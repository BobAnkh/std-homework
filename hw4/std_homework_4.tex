\PassOptionsToPackage{unicode=true}{hyperref} % options for packages loaded elsewhere
\PassOptionsToPackage{hyphens}{url}
%
\documentclass[]{article}
\usepackage[fontset=ubuntu]{ctex}
\usepackage[a4paper,scale=0.8]{geometry}
\usepackage{lmodern}
\usepackage{amssymb,amsmath}
\usepackage{ifxetex,ifluatex}
\usepackage{fixltx2e} % provides \textsubscript
\ifnum 0\ifxetex 1\fi\ifluatex 1\fi=0 % if pdftex
  \usepackage[T1]{fontenc}
  \usepackage[utf8]{inputenc}
  \usepackage{textcomp} % provides euro and other symbols
\else % if luatex or xelatex
  \usepackage{unicode-math}
  \defaultfontfeatures{Ligatures=TeX,Scale=MatchLowercase}
\fi
% use upquote if available, for straight quotes in verbatim environments
\IfFileExists{upquote.sty}{\usepackage{upquote}}{}
% use microtype if available
\IfFileExists{microtype.sty}{%
\usepackage[]{microtype}
\UseMicrotypeSet[protrusion]{basicmath} % disable protrusion for tt fonts
}{}
\IfFileExists{parskip.sty}{%
\usepackage{parskip}
}{% else
\setlength{\parindent}{2em}
\setlength{\parskip}{6pt plus 2pt minus 1pt}
}
\usepackage{hyperref}
\hypersetup{
            pdfborder={0 0 0},
            breaklinks=true}
\urlstyle{same}  % don't use monospace font for urls
\setlength{\emergencystretch}{3em}  % prevent overfull lines
\providecommand{\tightlist}{%
  \setlength{\itemsep}{0pt}\setlength{\parskip}{0pt}}
\setcounter{secnumdepth}{0}
% Redefines (sub)paragraphs to behave more like sections
\ifx\paragraph\undefined\else
\let\oldparagraph\paragraph
\renewcommand{\paragraph}[1]{\oldparagraph{#1}\mbox{}}
\fi
\ifx\subparagraph\undefined\else
\let\oldsubparagraph\subparagraph
\renewcommand{\subparagraph}[1]{\oldsubparagraph{#1}\mbox{}}
\fi
\usepackage{graphicx, subfigure}
\usepackage{enumerate}
\usepackage{booktabs}
\setlength{\parindent}{2em}
\pagestyle{plain}
% set default figure placement to htbp
\makeatletter
\def\fps@figure{htbp}
\makeatother
\title{视听信息系统导论\,第四次作业}
\author{BobAnkh}
\date{December 2020}

\begin{document}

\maketitle

\hypertarget{header-n80}{%
\subsection{1. 线性对数几率回归中假设$f(x)=w^T x+b$,设计一种数据处理方案使得参数b可以不用出现}\label{header-n80}}

    可以在向量$x$的末尾增加一个取值全为1的维度;或者也可以将所有的数据都减去某一个特定的数据,如$x_k-x_0$。

\hypertarget{header-n87}{%
\subsection{2. 用牛顿法求解无约束优化问题,目标函数为}\label{header-n87}}
$$
f(x)=\log ⁡(e^x+e^{-x})
$$

\textbf{分别以初值$x^{(0)}$=1和$x^{(0)}=1.1$,写出前五步迭代过程,令迭代步长=1.}
    
    目标函数为$f(x)=\log(e^x+e^{-x})$,则可以得到其一阶导和二阶导为:
    
    $$
    f^{'}(x)=\frac{e^x-e^{-x}}{e^x+e^{-x}}
    $$
    
    $$
    f^{''}(x)=\frac{4}{(e^x+e^{-x})^2}
    $$
    
    
    由此进行迭代:
    
\begin{enumerate}
    \item $x^{(0)}=1$时
    
    \begin{center}
    \begin{tabular}{|c|c|}
    \hline
        $x^{(0)}$ & 1 \\
        \hline
        $x^{(1)}$ & -0.8134 \\
        \hline
        $x^{(2)}$ & 0.4094 \\
        \hline
        $x^{(3)}$ & -0.04730 \\
        \hline
        $x^{(4)}$ & 7.060e-5\\
        \hline
        $x^{(5)}$ & -2.347e-13 \\
    \hline
    \end{tabular}
    \end{center}
    
    \item $x^{(0)}=1.1$时
    
    \begin{center}
    \begin{tabular}{|c|c|}
    \hline
        $x^{(0)}$ & 1.1 \\
        \hline
        $x^{(1)}$ & -1.129 \\
        \hline
        $x^{(2)}$ & 1.234 \\
        \hline
        $x^{(3)}$ & -1.695 \\
        \hline
        $x^{(4)}$ & 5.715\\
        \hline
        $x^{(5)}$ & -23021 \\
    \hline
    \end{tabular}
    \end{center}
    
\end{enumerate}

\hypertarget{header-n94}{%
\subsection{3. 总变分(total variation)是视听处理领域一种经典的去除噪声的方法。}\label{header-n94}}

\textbf{假设输入的被噪声污染的信号用列向量表示为$x\in R^n$,用$x_i$表示x ̂的第i个元素。则去噪之后的信号$x\in R^n$通过下面的无约束优化问题求得,其中$\epsilon>0$,$\mu>0$是两个常数。}

$$
minimize f(x)=\lVert x - \hat{x} \rVert_2^2 +\mu \sum_{i=1}^{n-1} \sqrt{\epsilon^2+(x_{i+1}-x_i)^2}-\epsilon)
$$

\textbf{1)若采用梯度下降法求解该问题,写出下降方向$\Delta x$的数学表达式(明确给出$\Delta x$每个元素的表达形式);}

\textbf{2)若采用牛顿法求解该问题,写出计算下降方向$\Delta x$的线性方程组的数学表达式(明确给出系数矩阵每个元素的表达形式);}

\textbf{3)分析求解问题(2)中的线性方程组的时间复杂度。}

\begin{enumerate}[(1)]
\item
    在梯度下降法中,$\Delta x=-\nabla f(x)$,故有:
    
    $$
    \frac{\partial f}{\partial x_1}=2(x_1-\hat{x_1})+\mu \frac{x_1-x_{2}}{\sqrt{\epsilon^2+(x_{2}-x_1)^2}}
    $$
    
    $$
    \frac{\partial f}{\partial x_i}=2(x_i-\hat{x_i})+\mu \frac{x_i-x_{i+1}}{\sqrt{\epsilon^2+(x_{i+1}-x_i)^2}}+\mu \frac{x_i-x_{i-1}}{\sqrt{\epsilon^2+(x_{i}-x_{i-1})^2}}\;,\quad for\;1<i<n
    $$
    
    $$
    \frac{\partial f}{\partial x_n}=2(x_n-\hat{x_n})+\mu \frac{x_n-x_{n-1}}{\sqrt{\epsilon^2+(x_{n}-x_{n-1})^2}}
    $$

\item
    在牛顿法中,$\nabla^2 f(x) \Delta x=-\nabla f(x)$,故$\Delta x$可通过解此方程求得。$\nabla f(x)$上一问中已经计算过,本问中计算海森矩阵$\nabla^2 f(x)$即可:
    
    $\nabla^2 f(x)$首行为$\frac{\partial^2 f(x)}{\partial x_1 \partial x_i}$,故可知该行第一个元素为$2+\mu \epsilon^2(\epsilon^2+(x_2-x_1)^2)^{-\frac{3}{2}}$,第二个元素为$-\mu \epsilon^2(\epsilon^2+(x_2-x_1)^2)^{-\frac{3}{2}}$,其余元素为0
    
    $\nabla^2 f(x)$中间第i行$(1<i<n)$为$\frac{\partial^2 f(x)}{\partial x_i \partial x_j}$($1\leq j \leq n$),故可知对于这第i行,第i-1个元素为$-\mu \epsilon^2(\epsilon^2+(x_i-x_{i-1})^2)^{-\frac{3}{2}}$,第i个元素为$2+\mu \epsilon^2(\epsilon^2+(x_{i+1}-x_{i})^2)^{-\frac{3}{2}}+\mu \epsilon^2(\epsilon^2+(x_i-x_{i-1})^2)^{-\frac{3}{2}}$,第i+1个元素为$-\mu \epsilon^2(\epsilon^2+(x_{i+1}-x_{i})^2)^{-\frac{3}{2}}$,其余元素为0
    
    $\nabla^2 f(x)$最后一行(即第n行)为$\frac{\partial^2 f(x)}{\partial x_1 \partial x_i}$,故可知该行第n-1个元素为$-\mu \epsilon^2(\epsilon^2+(x_{n}-x_{n-1})^2)^{-\frac{3}{2}}$,第n个元素为$2+\mu \epsilon^2(\epsilon^2+(x_{n}-x_{n-1})^2)^{-\frac{3}{2}}$,其余元素为0

\item
    由(2)中分析过程可知,对于本问题中的海森矩阵,除首尾两行各2个元素非0外,每行只有两个元素非0,故可直接使用高斯消元法求解,时间复杂度是O(n)级别的。

\end{enumerate}
\hypertarget{header-n102}{%
\subsection{4. x为二维非负向量, 证明:}\label{header-n102}}

\textbf{双曲集合$\{x\in R_+^2 | x_1 x_2≥1\}$是凸集合}

假设$(x_1,x_2)$和$(y_1,y_2)$都属于该双曲集合,那么对于$\theta\in [0,1]$,有:
$$
z_1=\theta x_1+(1-\theta)y_1\;,\;z_2=\theta x_2+(1-\theta)y_2
$$

若$x_1\geq y_1$且$x_2\geq y_2$,那么显然有$z_1\geq y_1\;,\;z_2\geq y_2$,则可得$z_1z_2\geq y_1y_2\geq 1$

若$y_1\geq x_1$且$y_2\geq x_2$,那么显然有$z_1\geq x_1\;,\;z_2\geq x_2$,则可得$z_1z_2\geq x_1x_2\geq 1$

而另外的两种情况必然会有$(y_1-x_1)(y_2-x_2)<0$,则可求得:
\begin{align*}
z_1z_2 &= (\theta x_1+(1-\theta)y_1)(\theta x_2+(1-\theta)y_2)\\
       & =\theta^2 x_1x_2+(1-\theta)^2 y_1y_2 + \theta(1-\theta)(y_1x_2+x_1y_2)\\
       & =\theta x_1x_2 + (1-\theta)y_1y_2-\theta(1-\theta)(y_1-x_1)(y_2-x_2)\\
       & \geq 1
\end{align*}

综上,可以证明该双曲集合是凸集合。

\hypertarget{header-n104}{%
\subsection{5. 已知$R^{m+n}$上的两个凸集合$S_1$和$S_2$}\label{header-n104}}

\textbf{证明它们的部分和集合S也是凸集合: $S=\{(x,y_1+y_2)|x\in R^m,y_1,y_2\in R^n,(x,y_1)\in S_1,(x,y_2)\in S_2\}$}

取$(\bar{x},\bar{y_1})\in S_1\;,\;(\bar{x},\bar{y_2})\in S_2\;,\;(\hat{x},\hat{y_1})\in S_1\;,\;(\hat{x},\hat{y_2})\in S_2$,则

$(\bar{x},\bar{y_1}+\bar{y_2})\in S\;,\;(\hat{x},\hat{y_1}+\hat{y_2})\in S$

对于$\theta\in[0,1]$,有$\theta (\bar{x},\bar{y_1}+\bar{y_2})+(1-\theta)(\hat{x},\hat{y_1}+\hat{y_2})=(\theta\bar{x}+(1-\theta)\hat{x}\,,\,(\theta\bar{y_1}+(1-\theta)\hat{y_1})+(\theta\bar{y_2}+(1-\theta)\hat{y_2}))$

而已知$S_1$和$S_2$都是凸集,故有$(\theta\bar{x}+(1-\theta)\hat{x},\theta\bar{y_1}+(1-\theta)\hat{y_1})\in S_1\;,\;(\theta\bar{x}+(1-\theta)\hat{x},\theta\bar{y_2}+(1-\theta)\hat{y_2})\in S_2$

所以可以得到$(\theta\bar{x}+(1-\theta)\hat{x}\,,\,(\theta\bar{y_1}+(1-\theta)\hat{y_1})+(\theta\bar{y_2}+(1-\theta)\hat{y_2}))\in S$,由此得证。

\hypertarget{header-n106}{%
\subsection{6. 连续函数$f:R^n→R$}\label{header-n106}}

\textbf{证明f为凸函数当且仅当对任意$x,y\in R^n$, 下式成立:$\int_0^1 f(x+\lambda(y-x)){\rm d}\lambda \leq \frac{f(x)+f(y)}{2}$}

先证f为凸函数时,该式成立:

根据Jensen's不等式,对于$\forall \lambda \in[0,1]$有
$$
f(\lambda y +(1-\lambda)x)\leq \lambda f(y) +(1-\lambda)f(x)
$$

两边在$[0,1]$上积分可得:
$$
\int_0^1 f(\lambda y +(1-\lambda)x) d\lambda \leq \int_0^1 (\lambda f(y) +(1-\lambda)f(x))d\lambda=\frac{f(x)+f(y)}{2}
$$

此即$\int_0^1 f(x +\lambda(y-x)) d\lambda \leq \frac{f(x)+f(y)}{2}$

再证f不是凸函数时,该式一定不成立:

若f不是凸函数,则Jensen's不等式不成立,即$\exists x,y,\lambda_0 \in (0,1)$,使得$f(\lambda_0x+(1-\lambda_0)y)>\lambda_0f(x)+(1-\lambda_0)f(y)$

考虑连续函数$F(\lambda)=f(\lambda x+(1-\lambda)y)-\lambda f(x)-(1-\lambda)f(y)$,显然仅当$\lambda=0$或$\lambda=1$时,$F(\lambda)=0$,对于$\lambda_0$,有$F(\lambda_0)>0$

若记$\alpha\;,\;\beta$分别为$\lambda_0$左侧和右侧最近的零点,则在区间$(\alpha,\beta)$上,有
$$
f(\lambda x +(1-\lambda)y)>\lambda f(x) + (1-\lambda)f(y)
$$

令$u=\alpha x +(1-\alpha)y\;,\;v=\beta x+(1-\beta)y$,则对于$\lambda \in (0,1)$有
$$
f(\lambda u +(1-\lambda)v)>\lambda f(u) +(1-\lambda)f(v)
$$

两边在$[0,1]$上积分可得:
$$
\int_0^1 f(v+\lambda (u-v))d\lambda>\int_0^1 (f(v)+\lambda(f(u)-f(v)))d\lambda=\frac{f(u)+f(v)}{2}
$$

这与题设条件相矛盾,由此可以说明在f不是凸函数时,该式一定不成立。

综上可知,当且仅当f为凸函数是,该式成立。

\subsection{7. 两个元素取值为正数的向量x, y满足:}

\textbf{$x,y\in R_{++}^n$,其相对熵定义如下,其中$x_k$, $y_k$分别是向量x,y的第k个元素,log表示自然对数:}

$$
\sum_{k=1}^n x_k \log (\frac{x_k}{y_k})
$$

\textbf{1) 证明相对熵是变量x,y的凸函数;}

\textbf{2) 假设y是一个给定的常数向量,定义如下的等式约束优化问题,其中$A\in R^{m\times n}$, $b\in R^m$:}
$$
minimize\;\sum_{k=1}^n x_k \log (\frac{x_k}{y_k})
$$
$$
subject\;to\;Ax=b
$$
$$
\sum_{k=1}^n x_k=1
$$

\textbf{证明该优化问题的对偶问题是如下形式,其中$a_k$是矩阵A的第k列:}
$$
minimize\;b^T z+\log \sum_{k=1}^n y_k e^{-a_k^T z}
$$

\begin{enumerate}
    \item 
    
    先证明相对熵是x的凸函数。记相对熵为$f(x,y)=\sum_{k=1}^n x_k \log\frac{x_k}{y_k}$
    
    令$g(x)=x\log\frac{x}{y}$,其中y此时是常数参量,则有$g^{''} (x)=\frac{1}{x}$,显然在$x>0$时,$g(x)$是凸函数
    
    对于$\forall \theta \in[0,1]$有
    \begin{align*}
    \theta f(x_1,y)+(1-\theta)f(x_2,y) &= \theta \sum_{k=1}^n x_{1k} \log\frac{x_{1k}}{y_k}+(1-\theta)\sum_{k=1}^n x_{2k} \log\frac{x_{2k}}{y_k}\\
    &= \sum_{k=1}^n(\theta x_{1k} \log\frac{x_{1k}}{y_k}+(1-\theta) x_{2k} \log\frac{x_{2k}}{y_k})\\
    &\geq \sum_{k=1}^n [\theta x_{1k}+(1-\theta) x_{2k}] \log\frac{[\theta x_{1k}+(1-\theta) x_{2k}]}{y_k}\\
    &= f(\theta x_1 + (1-\theta)x_2, y)
    \end{align*}

    由此可说明$f(x,y)$是x的凸函数。
    
    接下来证明相对熵是y的凸函数。此时可以将相对熵写作如下形式:
    $$
    f(x,y)=\sum_{k=1}^n x_k \log\frac{x_k}{y_k}=\sum_{k=1}^n x_k \log x_k-\sum_{k=1}^n x_k \log y_k
    $$
    
    令$h(y)=-x\log y$,其中x此时是常数参量,则有$h^{''} (y)=\frac{1}{y^2}$,显然在$y>0$时,$h(x)$是凸函数
    
    对于$\forall \theta \in[0,1]$有
    \begin{align*}
    \theta f(x,y_1)+(1-\theta)f(x,y_2) &= \sum_{k=1}^n (x_k \log x_k-\theta x_k \log y_{1k}-(1-\theta) x_k \log y_{2k})\\
    &\geq \sum_{k=1}^n (x_k \log x_k -x_k \log (\theta y_{1k}+(1-\theta)y_{2k}))\\
    &= f(x,\theta y_1 + (1-\theta)y_2)
    \end{align*}
    
    由此可说明$f(x,y)$是y的凸函数。
    
    \item
    
    对于等式约束优化问题,有$L(x, \lambda,\mu)=\sum_{k=1}^n x_k \log\frac{x_k}{y_k}+\lambda^T(Ax-b)+\mu(\sum_{k=1}^n x_k-1)$
    
    令$\frac{\partial L}{\partial x_k}=0\;,\;\frac{\partial L}{\partial \lambda}=0\;,\;\frac{\partial L}{\partial \mu}=0$,则有
    
    \begin{align}
        \log\frac{x_k}{y_k} &=-1-\mu-\lambda^T a_k \label{x-1}\\
        Ax-b &=0 \label{x-2}\\
        \sum_{k=1}^n x_k-1 &=0 \label{x-3}
    \end{align}

    将(\ref{x-1})(\ref{x-2})(\ref{x-3})三式代入$L(x,\lambda,\mu)$可得原问题的对偶形式:
$$
\min_x \sum_{k=1}^n x_k\log\frac{x_k}{y_k} \iff \max_{\lambda, \mu}g(\lambda, \mu)
$$

其中,$g(\lambda,\mu)$满足:
\begin{align*}
g(\lambda,\mu)=L(x,\lambda,\mu) &=\sum_{k=1}^n x_k (-1-\mu-\lambda^T a_k)\\
&=-\sum_{k=1}^n x_k a_K^T \lambda-(1+\mu)\\
&=-(AX)^T\lambda-(1+\mu)
\end{align*}

又由(\ref{x-1})知:$x_k=y_k e^{-1-\mu-\lambda^T a_k}$,由此可得$\mu+1=\log\sum_{k=1}^n y_k e^{-a_K^T\lambda}$

故$g(\lambda,\mu)=-b^T\lambda -\log\sum_{k=1}^n y_k e^{-a_K^T\lambda}$

由此可得,原问题等价于$\min_{\lambda}(-b^T\lambda -\log\sum_{k=1}^n y_k e^{-a_K^T\lambda})$,此即$\min_{\lambda}(b^T\lambda +\log\sum_{k=1}^n y_k e^{-a_K^T\lambda})$,证毕。

    
\end{enumerate}

\end{document}
